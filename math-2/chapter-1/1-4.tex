\documentclass[8pt,dvipdfmx]{article}[b5paper]
\usepackage[utf8]{inputenc}
\usepackage{amsmath}
\usepackage{amssymb}
\usepackage{xcolor}
\usepackage{tcolorbox}
\usepackage{geometry}
\usepackage{fancyhdr} % ヘッダーとフッターをカスタマイズするためのパッケージ

\geometry{top=25mm, headheight=15mm} % ヘッダーに十分なスペースを確保

\pagestyle{fancy}
\fancyhf{} % 既存のヘッダーとフッターをクリア
\fancyhead[C]{指数・対数関数(1)} % ヘッダーの中央にテキストを配置
\renewcommand{\headrulewidth}{2pt} % ヘッダーの下の線の太さ
\renewcommand{\footrulewidth}{0pt} % フッターの線は表示しない

\begin{document}

% 問題セクション
\begin{tcolorbox}[title=数学\textcircled{2} 1-4 BC]
\section*{問題}

\begin{itemize}
    \item[(1)] \( a = \dfrac{2}{3} \) に対して, \( b = a^a \) とするとき,\( 2^a a^a \) と \( 2^b a^b \) の大小を比較せよ。
    \item[(2)] \( a \)  を 1 でない正の定数とするとき, 不等式
    \[
    a^{2x} + a^{x-1} < a^{x+2} + a
    \]
    を解け.
\end{itemize}
\end{tcolorbox}


% 計算スペース
\end{document}
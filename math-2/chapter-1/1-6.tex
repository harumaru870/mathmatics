\documentclass[8pt,dvipdfmx]{article}[b5paper]
\usepackage[utf8]{inputenc}
\usepackage{amsmath}
\usepackage{amssymb}
\usepackage{enumerate}
\usepackage{xcolor}
\usepackage{tcolorbox}
\usepackage{geometry}
\usepackage{fancyhdr} % ヘッダーとフッターをカスタマイズするためのパッケージ

\geometry{top=25mm, headheight=15mm} % ヘッダーに十分なスペースを確保

\pagestyle{fancy}
\fancyhf{} % 既存のヘッダーとフッターをクリア
\fancyhead[C]{指数・対数関数(1)} % ヘッダーの中央にテキストを配置
\renewcommand{\headrulewidth}{2pt} % ヘッダーの下の線の太さ
\renewcommand{\footrulewidth}{0pt} % フッターの線は表示しない

\begin{document}

% 問題セクション
\begin{tcolorbox}[title=数学\textcircled{2} 1-6 C]
\( a \) を実数の定数とする. 関数 \( y = 4^{x} + 4^{-x} - a(2^{x+1} + 2^{-x+1}) \) について, 次の問に答えよ.
\begin{enumerate}[(1)]
    \item \( 2^{x}+2^ {- x}=t \)とするとき,\( y \) を \( t \) を用いて表せ.
    \item \( y \) の最小値を求めよ.
\end{enumerate}
\end{tcolorbox}




% 計算スペース
\end{document}
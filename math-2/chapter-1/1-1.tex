\documentclass[8pt,dvipdfmx]{article}[b5paper]
\usepackage[utf8]{inputenc}
\usepackage{amsmath}
\usepackage{xcolor}
\usepackage{tcolorbox}
\usepackage{geometry}
\usepackage{fancyhdr} % ヘッダーとフッターをカスタマイズするためのパッケージ

\geometry{top=25mm, headheight=15mm} % ヘッダーに十分なスペースを確保

\pagestyle{fancy}
\fancyhf{} % 既存のヘッダーとフッターをクリア
\fancyhead[C]{指数・対数関数(1)} % ヘッダーの中央にテキストを配置
\renewcommand{\headrulewidth}{2pt} % ヘッダーの下の線の太さ
\renewcommand{\footrulewidth}{0pt} % フッターの線は表示しない

\begin{document}

% 問題セクション
\begin{tcolorbox}[title=数学\textcircled{2} 1-1 AB]
\section*{問題}
次の式を計算せよ.

\begin{enumerate}
    \item[(1)] $\sqrt[3]{81} + \sqrt[3]{-27}$

    \vspace{2mm} % 問題間のスペース

    \item[(2)]$\sqrt[3]{a} \times \sqrt[6]{a} \div \sqrt{a} \quad (a \text{ は正の定数})$

    \vspace{2mm} % 問題間のスペース

    \item[(3)] $\frac{2}{3}^4 - 24 \times 18^3$
    
    \vspace{2mm} % 問題間のスペース

    \item[(4)] $8^{-\frac{2}{3}} + \left(\frac{1}{81}\right)^{-\frac{1}{4}}$

\end{enumerate}
\end{tcolorbox}

% 計算スペース
\end{document}
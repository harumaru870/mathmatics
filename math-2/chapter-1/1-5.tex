\documentclass[8pt,dvipdfmx]{article}[b5paper]
\usepackage[utf8]{inputenc}
\usepackage{amsmath}
\usepackage{amssymb}
\usepackage{xcolor}
\usepackage{tcolorbox}
\usepackage{geometry}
\usepackage{fancyhdr} % ヘッダーとフッターをカスタマイズするためのパッケージ

\geometry{top=25mm, headheight=15mm} % ヘッダーに十分なスペースを確保

\pagestyle{fancy}
\fancyhf{} % 既存のヘッダーとフッターをクリア
\fancyhead[C]{指数・対数関数(1)} % ヘッダーの中央にテキストを配置
\renewcommand{\headrulewidth}{2pt} % ヘッダーの下の線の太さ
\renewcommand{\footrulewidth}{0pt} % フッターの線は表示しない

\begin{document}

% 問題セクション
\begin{tcolorbox}[title=数学\textcircled{2} 1-5 AB]
関数 \( f(x) = 9^{x} -9^{- x} + 2\cdot3^{x}+ 2\cdot3~{-x} \) について, 次の問に答えよ.
\begin{enumerate}
    \item[(1)] \( f(t) = 3^{x}+3^{-x} \) とおくとき, \( f(x) \) を \( t \) を用いて表せ.
    \item[(2)] \( f(x) \) の最小値とそのときの \( x \) の値を求めよ.
\end{enumerate}
\end{tcolorbox}



% 計算スペース
\end{document}
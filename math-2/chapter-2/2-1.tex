\documentclass[8pt,dvipdfmx]{article}[b5paper]
\usepackage[utf8]{inputenc}
\usepackage{amsmath}
\usepackage{amssymb}
\usepackage{enumerate}
\usepackage{xcolor}
\usepackage{tcolorbox}
\usepackage{geometry}
\usepackage{fancyhdr} % ヘッダーとフッターをカスタマイズするためのパッケージ

\geometry{top=25mm, headheight=15mm} % ヘッダーに十分なスペースを確保

\pagestyle{fancy}
\fancyhf{} % 既存のヘッダーとフッターをクリア
\fancyhead[C]{指数・対数関数(2)} % ヘッダーの中央にテキストを配置
\renewcommand{\headrulewidth}{2pt} % ヘッダーの下の線の太さ
\renewcommand{\footrulewidth}{0pt} % フッターの線は表示しない

\begin{document}

% 問題セクション
\begin{tcolorbox}[title=数学\textcircled{2} 2-1 A]
次の式を計算せよ.
\begin{enumerate}[(1)]
    \item \( \log_6 24 + \log_6 3 - \log_6 2 \)
    \item \( (\log_2 9 + \log_4 3)(\log_3 2 + \log_9 4) \)
\item $4^{\log_2 3}$
\end{enumerate}
\end{tcolorbox}




% 計算スペース
\end{document}
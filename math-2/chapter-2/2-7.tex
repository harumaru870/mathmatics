\documentclass[8pt,dvipdfmx]{article}[b5paper]
\usepackage[utf8]{inputenc}
\usepackage{amsmath}
\usepackage{amssymb}
\usepackage{enumerate}
\usepackage{xcolor}
\usepackage{tcolorbox}
\usepackage{geometry}
\usepackage{fancyhdr} % ヘッダーとフッターをカスタマイズするためのパッケージ

\geometry{top=25mm, headheight=15mm} % ヘッダーに十分なスペースを確保

\pagestyle{fancy}
\fancyhf{} % 既存のヘッダーとフッターをクリア
\fancyhead[C]{指数・対数関数(2)} % ヘッダーの中央にテキストを配置
\renewcommand{\headrulewidth}{2pt} % ヘッダーの下の線の太さ
\renewcommand{\footrulewidth}{0pt} % フッターの線は表示しない

\begin{document}

% 問題セクション
\begin{tcolorbox}[title=数学\textcircled{\scriptsize 2} 2-7 C]
 $\log_{10}2 = 0.3010$,$\log_{10}3 = 0.4771$ とする.
\begin{enumerate}[(1)]
   \item  $n$を正の整数とする。$15^n$が45桁の整数となるような$n$を求めよ。さらにこのとき、$15^n$の最高位の数字を求めよ。
    \item$15^{-20}$を小数で表したとき、小数第何位に初めて0でない数字が現れるか。また、その数字は何か。
\end{enumerate}
\end{tcolorbox}




% 計算スペース
\end{document}
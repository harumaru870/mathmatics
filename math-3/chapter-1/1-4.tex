\documentclass[8pt,dvipdfmx]{article}[b5paper]
\usepackage[utf8]{inputenc}
\usepackage{amsmath}
\usepackage{amssymb}
\usepackage{enumerate}
\usepackage{xcolor}
\usepackage{tcolorbox}
\usepackage{geometry}
\usepackage{fancyhdr} % ヘッダーとフッターをカスタマイズするためのパッケージ

\geometry{top=25mm, headheight=15mm} % ヘッダーに十分なスペースを確保

\pagestyle{fancy}
\fancyhf{} % 既存のヘッダーとフッターをクリア
\fancyhead[C]{三角関数(1)} % ヘッダーの中央にテキストを配置
\renewcommand{\headrulewidth}{2pt} % ヘッダーの下の線の太さ
\renewcommand{\footrulewidth}{0pt} % フッターの線は表示しない

\begin{document}

% 問題セクション
\begin{tcolorbox}[title=数学\textcircled{\scriptsize 3} 1-4 AB]
  \begin{enumerate}[(1)]
    \item$\tan \theta = 2$ のとき、$\cos^2 \theta$,$\sin \theta \cos \theta$ の値をそれぞれ求めよ。
        \item$\sin \theta + \cos \theta = -\dfrac{1}{2} \quad (0 \leq \theta \leq \pi)$ のとき、$\sin \theta \cos \theta$,$\cos \theta - \sin \theta$ の値をそれぞれ求めよ。
  \end{enumerate}
\end{tcolorbox}
% 計算スペース
\end{document}
\documentclass[8pt,dvipdfmx]{article}[b5paper]
\usepackage[utf8]{inputenc}
\usepackage{amsmath}
\usepackage{amssymb}
\usepackage{enumerate}
\usepackage{xcolor}
\usepackage{tcolorbox}
\usepackage{geometry}
\usepackage{fancyhdr} % ヘッダーとフッターをカスタマイズするためのパッケージ

\geometry{top=25mm, headheight=15mm} % ヘッダーに十分なスペースを確保

\pagestyle{fancy}
\fancyhf{} % 既存のヘッダーとフッターをクリア
\fancyhead[C]{三角関数(2)} % ヘッダーの中央にテキストを配置
\renewcommand{\headrulewidth}{2pt} % ヘッダーの下の線の太さ
\renewcommand{\footrulewidth}{0pt} % フッターの線は表示しない

\begin{document}

% 問題セクション
\begin{tcolorbox}[title=数学\textcircled{\scriptsize 3} 2-3 AB]
関数 \[f(\theta)=\sqrt{3}\sin\theta+\cos\theta\] について、
\begin{enumerate}[(1)]
\item $0\leqq\theta<2\pi$ における最大値と最小値をそれぞれ求めよ。
\item $0\leqq\theta\leqq\dfrac{\pi}{2}$ における最大値と最小値をそれぞれ求めよ。
\end{enumerate}
\end{tcolorbox}




% 計算スペース
\end{document}
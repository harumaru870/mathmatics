\documentclass[8pt,dvipdfmx]{article}[b5paper]
\usepackage[utf8]{inputenc}
\usepackage{amsmath}
\usepackage{amssymb}
\usepackage{enumerate}
\usepackage{xcolor}
\usepackage{tcolorbox}
\usepackage{geometry}
\usepackage{fancyhdr} % ヘッダーとフッターをカスタマイズするためのパッケージ

\geometry{top=25mm, headheight=15mm} % ヘッダーに十分なスペースを確保

\pagestyle{fancy}
\fancyhf{} % 既存のヘッダーとフッターをクリア
\fancyhead[C]{三角関数(2)} % ヘッダーの中央にテキストを配置
\renewcommand{\headrulewidth}{2pt} % ヘッダーの下の線の太さ
\renewcommand{\footrulewidth}{0pt} % フッターの線は表示しない

\begin{document}

% 問題セクション
\begin{tcolorbox}[title=数学\textcircled{\scriptsize 3} 2-4 C]
$c$は実数の定数とする。$\theta$の方程式
\[
\sin\theta+\sqrt{3}\cos\theta=c
\]
が$-\pi<\theta<\pi$の範囲に異なる2個$\alpha,\beta$をもつとする。
\begin{enumerate}[(1)]
\item $c$のとり得る値の範囲を求めよ。
\item $\tan\dfrac{\theta}{2}=t$とおくとき、$\sin\theta,\cos\theta$を$t$を用いて表せ。
\item $\tan\dfrac{\alpha+\beta}{2}$の値を求めよ。
\end{enumerate}
\end{tcolorbox}




% 計算スペース
\end{document}
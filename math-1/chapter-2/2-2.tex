\documentclass[8pt,dvipdfmx]{article}
\usepackage[utf8]{inputenc}
\usepackage{amsmath}
\usepackage{amssymb}
\usepackage{xcolor}
\usepackage{tcolorbox}
\usepackage{geometry}
\usepackage{empheq}
\usepackage{okumacro}
\usepackage{pifont}
\usepackage{fancyhdr} % ヘッダーとフッターをカスタマイズするためのパッケージ

\geometry{top=25mm, headheight=15mm} % ヘッダーに十分なスペースを確保

\pagestyle{fancy}
\fancyhf{} % 既存のヘッダーとフッターをクリア
\fancyhead[C]{関数と方程式(2)} % ヘッダーの中央にテキストを配置
\renewcommand{\headrulewidth}{2pt} % ヘッダーの下の線の太さ
\renewcommand{\footrulewidth}{0pt} % フッターの線は表示しない

\begin{document}

% 問題セクション
\begin{tcolorbox}[title=数学\textcircled{1} 2- 2 B ]
\(a\)を実数の定数とする.\(x\)に関する2つの不等式
\begin{align*}
&x^2+x-6 > 0  \tag*{・・・\MARU{1}}\\
&x^2 -(a+1)x + a < 0  \tag*{・・・\MARU{2}}
\end{align*}
について次の問に答えよ.
\begin{enumerate}
    \item[(1)] \MARU{1}を解け.
    \vspace{2mm} % 問題間のスペース
    \item[(2)] \MARU{2}を解け.
    \vspace{2mm}
    \item[(3)]\MARU{1},\MARU{2}をともに満たす整数\(x\)がただ1つ存在するような\(a\)の値の範囲を求めよ.
\end{enumerate}
\end{tcolorbox}



% 計算スペース
\end{document}

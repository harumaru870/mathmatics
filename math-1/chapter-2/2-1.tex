\documentclass[8pt,dvipdfmx]{article}
\usepackage[utf8]{inputenc}
\usepackage{amsmath}
\usepackage{amssymb}
\usepackage{xcolor}
\usepackage{tcolorbox}
\usepackage{geometry}
\usepackage{fancyhdr} % ヘッダーとフッターをカスタマイズするためのパッケージ

\geometry{top=25mm, headheight=15mm} % ヘッダーに十分なスペースを確保

\pagestyle{fancy}
\fancyhf{} % 既存のヘッダーとフッターをクリア
\fancyhead[C]{関数と方程式(2)} % ヘッダーの中央にテキストを配置
\renewcommand{\headrulewidth}{2pt} % ヘッダーの下の線の太さ
\renewcommand{\footrulewidth}{0pt} % フッターの線は表示しない

\begin{document}

% 問題セクション
\begin{tcolorbox}[title=数学\textcircled{1} 2-1 A]
次の不等式、連立不等式をそれぞれ解け。
\begin{enumerate}
\item[(1)] $-12-9x < 2x-1 \leqq \frac{5x+7}{3}$

\vspace{2mm} % 問題間のスペース
\item[(2)] $x^2 + 4x - 5 \geqq 0$

\vspace{2mm} % 問題間のスペース
\item[(3)] $x^2 - 4x + 4 > 0$
\vspace{2mm} % 問題間のスペース
\item[(4)]
\begin{align*}
\begin{cases}
x^2 -3x \leqq 0 ,\\
x^2 -2x -1 > 0
\end{cases}
\end{align*}
\end{enumerate}
\end{tcolorbox}



% 計算スペース
\end{document}

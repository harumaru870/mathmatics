\documentclass[8pt,dvipdfmx]{article}
\usepackage[utf8]{inputenc}
\usepackage{amsmath}
\usepackage{amssymb}
\usepackage{xcolor}
\usepackage{tcolorbox}
\usepackage{geometry}
\usepackage{fancyhdr} % ヘッダーとフッターをカスタマイズするためのパッケージ

\geometry{top=25mm, headheight=15mm} % ヘッダーに十分なスペースを確保

\pagestyle{fancy}
\fancyhf{} % 既存のヘッダーとフッターをクリア
\fancyhead[C]{関数と方程式(1)} % ヘッダーの中央にテキストを配置
\renewcommand{\headrulewidth}{2pt} % ヘッダーの下の線の太さ
\renewcommand{\footrulewidth}{0pt} % フッターの線は表示しない

\begin{document}

% 問題セクション
\begin{tcolorbox}[title=数学\textcircled{1} 1-4 A]
aは実数の定数とする.関数
\[
f(x) = x^2 - 2ax + 4a + 5
\]
について
\begin{enumerate}
    \item[(1)] \(0 \leqq x \leqq 1\) における \(f(x)\) の最小値\(m(a)\)を求めよ.
    \vspace{2mm} % 問題間のスペース
    \item[(2)] \(0 \leqq x \leqq 1\) における \(g(x)\) の最大値\(M(a)\)を求めよ.
\end{enumerate}
\end{tcolorbox}

% 計算スペース
\end{document}

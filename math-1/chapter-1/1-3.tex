\documentclass[8pt,dvipdfmx]{article}
\usepackage[utf8]{inputenc}
\usepackage{amsmath}
\usepackage{amssymb}
\usepackage{xcolor}
\usepackage{tcolorbox}
\usepackage{geometry}
\usepackage{fancyhdr} % ヘッダーとフッターをカスタマイズするためのパッケージ

\geometry{top=25mm, headheight=15mm} % ヘッダーに十分なスペースを確保

\pagestyle{fancy}
\fancyhf{} % 既存のヘッダーとフッターをクリア
\fancyhead[C]{関数と方程式(1)} % ヘッダーの中央にテキストを配置
\renewcommand{\headrulewidth}{2pt} % ヘッダーの下の線の太さ
\renewcommand{\footrulewidth}{0pt} % フッターの線は表示しない

\begin{document}

% 問題セクション
\begin{tcolorbox}[title=数学\textcircled{1} 1-3 AB]
\begin{align*}
f(x) &= 2x^2 - 8x + 4, \\
g(x) &= -\left(2x^2 - 8x + 4\right)^2 - 2(2x^2 - 8x + 4) + 1
\end{align*}
がある.
\begin{enumerate}
    \item[(1)] \(0 \leqq x \leqq 3\) における \(f(x)\) の最大値と最小値を求めよ。
    \vspace{2mm} % 問題間のスペース
    \item[(2)] \(0 \leqq x \leqq 3\) における \(g(x)\) の最大値と最小値を求めよ。
\end{enumerate}
\end{tcolorbox}

% 計算スペース
\end{document}

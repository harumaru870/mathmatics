\documentclass[8pt,dvipdfmx]{article}
\usepackage[utf8]{inputenc}
\usepackage{amsmath}
\usepackage{amssymb}
\usepackage{xcolor}
\usepackage{tcolorbox}
\usepackage{geometry}
\usepackage{fancyhdr} % ヘッダーとフッターをカスタマイズするためのパッケージ

\geometry{top=25mm, headheight=15mm} % ヘッダーに十分なスペースを確保

\pagestyle{fancy}
\fancyhf{} % 既存のヘッダーとフッターをクリア
\fancyhead[C]{関数と方程式(1)} % ヘッダーの中央にテキストを配置
\renewcommand{\headrulewidth}{2pt} % ヘッダーの下の線の太さ
\renewcommand{\footrulewidth}{0pt} % フッターの線は表示しない

\begin{document}

% 問題セクション
\begin{tcolorbox}[title=数学\textcircled{1} 1-5 BC]
\begin{enumerate}
    \item[(1)] 関数 \[ z = xy \] がある.実数 \(x, y\) が \(x + 2y = 4\) を満たして動くとき、\(z\) の最大値を求めよ。
    \vspace{2mm} % 問題間のスペース

    \item[(2)] 関数 \[ z = x^2 + xy - 4y^2 \] がある.実数 \(x, y\) が \(x \geqq 0\), \(y \geqq 0\) で \(x + 2y = 4\) を満たして動くとき、\(z\) の最大値と最小値をそれぞれ求めよ。
\end{enumerate}
\end{tcolorbox}

% 計算スペース
\end{document}

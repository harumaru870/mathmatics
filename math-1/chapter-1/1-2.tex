\documentclass[8pt,dvipdfmx]{article}
\usepackage[utf8]{inputenc}
\usepackage{amsmath}
\usepackage{xcolor}
\usepackage{tcolorbox}
\usepackage{geometry}
\usepackage{fancyhdr} % ヘッダーとフッターをカスタマイズするためのパッケージ

\geometry{top=25mm, headheight=15mm} % ヘッダーに十分なスペースを確保

\pagestyle{fancy}
\fancyhf{} % 既存のヘッダーとフッターをクリア
\fancyhead[C]{関数と方程式(1)} % ヘッダーの中央にテキストを配置
\renewcommand{\headrulewidth}{2pt} % ヘッダーの下の線の太さ
\renewcommand{\footrulewidth}{0pt} % フッターの線は表示しない

\begin{document}

% 問題セクション
\begin{tcolorbox}[title=数学\textcircled{1} 1-2 AB]
\section*{問題}

頂点の座標が \((1,2)\) で、点 \((2,3)\) を通る放物線を \(y = f(x)\) とし、3点 \((0,13)\), \((1,8)\), \((2,5)\) を通る放物線を \(y = g(x)\) とする。
\begin{enumerate}
    \item[(1)] \(f(x)\) と \(g(x)\) をそれぞれ求めよ。
    \vspace{2mm} % 問題間のスペース
    \item[(2)] \(y = g(x)\) を \(x\) 軸方向に \(p\)、\(y\) 軸方向に \(q\) だけ平行移動すると、放物線 \(y = f(x)\) に一致するとき、\(p, q\) の値を求めよ。
\end{enumerate}
\end{tcolorbox}
% 計算スペース
\end{document}

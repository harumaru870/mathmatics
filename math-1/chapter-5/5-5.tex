\documentclass[8pt,dvipdfmx]{article}[b5paper]
\usepackage[utf8]{inputenc}
\usepackage{amsmath}
\usepackage{amssymb}
\usepackage{enumerate}
\usepackage{xcolor}
\usepackage{tcolorbox}
\usepackage{geometry}
\usepackage{fancyhdr} % ヘッダーとフッターをカスタマイズするためのパッケージ

\geometry{top=25mm, headheight=15mm}

\pagestyle{fancy}
\fancyhf{}
\fancyhead[C]{図形と方程式(2)}
\renewcommand{\headrulewidth}{2pt}
\renewcommand{\footrulewidth}{0pt}

\begin{document}

\begin{tcolorbox}[title=数学\textcircled{1} 5-5 C]
xy平面上に、2直線
\begin{align*}
    l&: (1 - k)x + (k + 1)y + k - 1 = 0, \\
    m&: x + ky + 1 = 0
\end{align*}
がある。$k$ がすべての実数をとるとき、$l$ と $m$ の交点の軌跡を求め、図示せよ。
\end{tcolorbox}
\end{document}

\documentclass[8pt,dvipdfmx]{article}[b5paper]
\usepackage[utf8]{inputenc}
\usepackage{amsmath}
\usepackage{amssymb}
\usepackage{enumerate}
\usepackage{xcolor}
\usepackage{tcolorbox}
\usepackage{geometry}
\usepackage{fancyhdr} % ヘッダーとフッターをカスタマイズするためのパッケージ

\geometry{top=25mm, headheight=15mm}

\pagestyle{fancy}
\fancyhf{}
\fancyhead[C]{図形と方程式(2)}
\renewcommand{\headrulewidth}{2pt}
\renewcommand{\footrulewidth}{0pt}

\begin{document}

\begin{tcolorbox}[title=数学\textcircled{1} 5-4 ABC]
xy平面上で、放物線 $C: y = x^2 - 2x + 3$ と、点 $(3, 2)$ を通る傾き $m$ の直線 $l$ が異なる2点 $P, Q$ で交わっている。
\begin{enumerate}[(1)]
\item $m$ のとり得る範囲を求めよ。
    \item $m$ が (1) の範囲で変化するとき、線分 $PQ$ の中点 $R$ の軌跡を求めよ。
\end{enumerate}
\end{tcolorbox}
\end{document}

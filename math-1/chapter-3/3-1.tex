\documentclass[8pt,dvipdfmx]{article}
\usepackage[utf8]{inputenc}
\usepackage{amsmath}
\usepackage{amssymb}
\usepackage{xcolor}
\usepackage{tcolorbox}
\usepackage{geometry}
\usepackage{fancyhdr} % ヘッダーとフッターをカスタマイズするためのパッケージ

\geometry{top=25mm, headheight=15mm} % ヘッダーに十分なスペースを確保

\pagestyle{fancy}
\fancyhf{} % 既存のヘッダーとフッターをクリア
\fancyhead[C]{関数と方程式(3)} % ヘッダーの中央にテキストを配置
\renewcommand{\headrulewidth}{2pt} % ヘッダーの下の線の太さ
\renewcommand{\footrulewidth}{0pt} % フッターの線は表示しない

\begin{document}

% 問題セクション
\begin{tcolorbox}[title=数学\textcircled{1} 3-1 AB]
\begin{enumerate}
  \item[(1)] 2次方程式 \[x^2-x-8=0\]
の2解を \(\alpha, \beta\) とするとき, \(\alpha^2+\beta^2\), \(\alpha^3+\beta^3\), \(\displaystyle\frac{\alpha^2}{\beta}+\frac{\beta^2}{\alpha}\) の値をそれぞれ求めよ.
    \vspace{2mm} % 問題間のスペース

\item[(2)]  3次方程式 
\[
x^3 - x^2 + 2x - 3 = 0
\]
の3解を \(\alpha, \beta, \gamma\) とするとき, \(\displaystyle \frac{1}{\alpha} + \frac{1}{\beta} + \frac{1}{\gamma}\) および \(\alpha^2 + \beta^2 + \gamma^2\) の値をそれぞれ求めよ.

\end{enumerate}
\end{tcolorbox}

% 計算スペース
\end{document}

\documentclass[8pt,dvipdfmx]{article}
\usepackage[utf8]{inputenc}
\usepackage{amsmath}
\usepackage{amssymb}
\usepackage{xcolor}
\usepackage{tcolorbox}
\usepackage{geometry}
\usepackage{empheq}
\usepackage{okumacro}
\usepackage{pifont}
\usepackage{fancyhdr}
\usepackage{ellipsis} % 追加: ellipsis パッケージ

\geometry{top=25mm, headheight=15mm}

\pagestyle{fancy}
\fancyhf{}
\fancyhead[C]{関数と方程式(3)}
\renewcommand{\headrulewidth}{2pt}
\renewcommand{\footrulewidth}{0pt}

\begin{document}

\begin{tcolorbox}[title=数学\textcircled{1} 3- 6 C]
\(a\) は \(a \neq 0\) を満たす実数の定数とする。\(x, y\) の連立方程式
\begin{align*}
\begin{cases}
y=ax(1-x) ,\\
y=ay(1-y)
\end{cases}
\end{align*}
が \(x \neq y\) を満たす実数の組 \((x, y)\) を解に持つような \(a\) の値の範囲を求めよ.
\end{tcolorbox}

\end{document}

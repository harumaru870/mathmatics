\documentclass[8pt,dvipdfmx]{article}
\usepackage[utf8]{inputenc}
\usepackage{amsmath}
\usepackage{amssymb}
\usepackage{xcolor}
\usepackage{tcolorbox}
\usepackage{geometry}
\usepackage{empheq}
\usepackage{okumacro}
\usepackage{pifont}
\usepackage{fancyhdr}
\usepackage{ellipsis} % 追加: ellipsis パッケージ

\geometry{top=25mm, headheight=15mm}

\pagestyle{fancy}
\fancyhf{}
\fancyhead[C]{関数と方程式(3)}
\renewcommand{\headrulewidth}{2pt}
\renewcommand{\footrulewidth}{0pt}

\begin{document}

\begin{tcolorbox}[title=数学\textcircled{1} 3- 5 ABC]
\(a,b\) は実数の定数とする.3次方程式
\[
x^3 + ax^2 + bx + 10 = 0 
\]
が $1+3i$ を解にもつとき,\(a, b\)の値,および他の解をすべて求めよ.
\end{tcolorbox}

\end{document}

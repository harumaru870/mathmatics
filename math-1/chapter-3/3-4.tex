\documentclass[8pt,dvipdfmx]{article}
\usepackage[utf8]{inputenc}
\usepackage{amsmath}
\usepackage{amssymb}
\usepackage{xcolor}
\usepackage{tcolorbox}
\usepackage{geometry}
\usepackage{empheq}
\usepackage{okumacro}
\usepackage{pifont}
\usepackage{fancyhdr}
\usepackage{ellipsis} % 追加: ellipsis パッケージ

\geometry{top=25mm, headheight=15mm}

\pagestyle{fancy}
\fancyhf{}
\fancyhead[C]{関数と方程式(3)}
\renewcommand{\headrulewidth}{2pt}
\renewcommand{\footrulewidth}{0pt}

\begin{document}

\begin{tcolorbox}[title=数学\textcircled{1} 3- 4 AB]
\(a,b\) は実数の定数とする.\(x\) の3次方程式
\[
x^3 + (a-1)x^2 + bx + 2a = 0\tag*{\dotso($*$)}\\ 
\]
は \(x=2\) を解にもつ。
\begin{enumerate}
    \item[(1)] $b$を$a$を用いて表せ.
    \vspace{2mm}
    \item[(2)]({$*$})が異なる3つの実数解をもつような$a$の値の範囲を求めよ.
    \vspace{2mm}
    \item[(3)]({$*$})が(実数の)重解をもつような$a$の値を求めよ.
\end{enumerate}
\end{tcolorbox}

\end{document}

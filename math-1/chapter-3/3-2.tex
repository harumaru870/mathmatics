\documentclass[8pt,dvipdfmx]{article}
\usepackage[utf8]{inputenc}
\usepackage{amsmath}
\usepackage{amssymb}
\usepackage{xcolor}
\usepackage{tcolorbox}
\usepackage{geometry}
\usepackage{fancyhdr} % ヘッダーとフッターをカスタマイズするためのパッケージ

\geometry{top=25mm, headheight=15mm} % ヘッダーに十分なスペースを確保

\pagestyle{fancy}
\fancyhf{} % 既存のヘッダーとフッターをクリア
\fancyhead[C]{関数と方程式(3)} % ヘッダーの中央にテキストを配置
\renewcommand{\headrulewidth}{2pt} % ヘッダーの下の線の太さ
\renewcommand{\footrulewidth}{0pt} % フッターの線は表示しない

\begin{document}

% 問題セクション
\begin{tcolorbox}[title=数学\textcircled{1} 3-2 A]
\begin{enumerate}
\item[(1)]次の連立方程式を解け.
\begin{align*}
\begin{cases}
x+y=7 ,\\
xy=5
\end{cases}
\end{align*}
\vspace{2mm} % 問題間のスペース

\item[(2)]\(x, y\) の連立方程式
\begin{align*}
\begin{cases}
x+y=k ,\\
xy=2
\end{cases}
\end{align*}
が,実数の組\((x,y)\)を解にもつような実数$k$の値の範囲を求めよ.
\end{enumerate}
\end{tcolorbox}



% 計算スペース
\end{document}

\documentclass[8pt,dvipdfmx]{article}[b5paper]
\usepackage[utf8]{inputenc}
\usepackage{amsmath}
\usepackage{amssymb}
\usepackage{enumerate}
\usepackage{xcolor}
\usepackage{tcolorbox}
\usepackage{geometry}
\usepackage{fancyhdr} % ヘッダーとフッターをカスタマイズするためのパッケージ

\geometry{top=25mm, headheight=15mm}

\pagestyle{fancy}
\fancyhf{}
\fancyhead[C]{図形と方程式(1)}
\renewcommand{\headrulewidth}{2pt}
\renewcommand{\footrulewidth}{0pt}

\begin{document}

\begin{tcolorbox}[title=数学\textcircled{1} 4- 1 A]
Oを原点とする\(x, y\)平面上に,2点A(\(1, 3\)) B(\(6, -2\))がある。
\begin{enumerate}[(1)]
    \item 線分ABを2:3に内分する点をP,1:2に外分する点をQとし,三角形OABの重心をGとする.P,Q,Gの座標をそれぞれ答えよ.
    \item 線分ABの長さを求めよ.
    \item 2点A,Bを通る直瀬$l$の方程式を求めよ.また,Gと$l$の距離を求めよ.
    \item 三角形GPQの面積を求めよ.
\end{enumerate}
\end{tcolorbox}

\end{document}

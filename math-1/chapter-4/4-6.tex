\documentclass[8pt,dvipdfmx]{article}[b5paper]
\usepackage[utf8]{inputenc}
\usepackage{amsmath}
\usepackage{amssymb}
\usepackage{enumerate}
\usepackage{xcolor}
\usepackage{tcolorbox}
\usepackage{geometry}
\usepackage{fancyhdr} % ヘッダーとフッターをカスタマイズするためのパッケージ

\geometry{top=25mm, headheight=15mm}

\pagestyle{fancy}
\fancyhf{}
\fancyhead[C]{図形と方程式(1)}
\renewcommand{\headrulewidth}{2pt}
\renewcommand{\footrulewidth}{0pt}

\begin{document}

\begin{tcolorbox}[title=数学\textcircled{1} 4-6 C]
\(x, y\)平面上に
\[
\text{円}C: x^2 + y^2 - 4x + 8y = 0
\]
がある.また, 2点A\((1,0)\), B\((5,-4)\)を通る直線を$l$とする.
\begin{enumerate}[(1)]
    \item 直線$l$と円$C$は共有点を持たないことを示せ.
    \item 点Pが円$C$上を動く時, 三角形ABPの面積の最大値およびその時のPの座標を求めよ.
\end{enumerate}
\end{tcolorbox}



\end{document}

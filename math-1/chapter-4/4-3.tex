\documentclass[8pt,dvipdfmx]{article}[b5paper]
\usepackage[utf8]{inputenc}
\usepackage{amsmath}
\usepackage{amssymb}
\usepackage{enumerate}
\usepackage{xcolor}
\usepackage{tcolorbox}
\usepackage{geometry}
\usepackage{fancyhdr} % ヘッダーとフッターをカスタマイズするためのパッケージ

\geometry{top=25mm, headheight=15mm}

\pagestyle{fancy}
\fancyhf{}
\fancyhead[C]{図形と方程式(1)}
\renewcommand{\headrulewidth}{2pt}
\renewcommand{\footrulewidth}{0pt}

\begin{document}

\begin{tcolorbox}[title=数学\textcircled{1} 4- 3 BC]
\(x,y\)平面上の3直線
\begin{align*}
3x-4y-1=0, \quad
2x-y-8=0 ,\quad
2x-ay+3=0,\quad
\end{align*}
によって三角形ができるような$a$の値の範囲を求めよ.
\end{tcolorbox}

\end{document}

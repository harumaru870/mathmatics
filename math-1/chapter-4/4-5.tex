\documentclass[8pt,dvipdfmx]{article}[b5paper]
\usepackage[utf8]{inputenc}
\usepackage{amsmath}
\usepackage{amssymb}
\usepackage{enumerate}
\usepackage{xcolor}
\usepackage{tcolorbox}
\usepackage{geometry}
\usepackage{fancyhdr} % ヘッダーとフッターをカスタマイズするためのパッケージ

\geometry{top=25mm, headheight=15mm}

\pagestyle{fancy}
\fancyhf{}
\fancyhead[C]{図形と方程式(1)}
\renewcommand{\headrulewidth}{2pt}
\renewcommand{\footrulewidth}{0pt}

\begin{document}

\begin{tcolorbox}[title=数学\textcircled{1} 4-5 ABC]
\(x, y\)平面上に円$C$:\(x^2+y^2-4x+8y=0\)がある.
\begin{enumerate}[(1)]
    \item 点\((4,2)\)から$C$に引いた接線の方程式をすべて求めよ.
    \item 点\((4,2)\)を通る直線$l$が$C$と異なる2点で交わり,その2点間の距離が$4\sqrt{3}$であるとき,$l$の方程式をすべて求めよ.
\end{enumerate}
\end{tcolorbox}


\end{document}

\documentclass[8pt,dvipdfmx]{article}[b5paper]
\usepackage[utf8]{inputenc}
\usepackage{amsmath}
\usepackage{amssymb}
\usepackage{enumerate}
\usepackage{xcolor}
\usepackage{tcolorbox}
\usepackage{geometry}
\usepackage{fancyhdr} % ヘッダーとフッターをカスタマイズするためのパッケージ

\geometry{top=25mm, headheight=15mm}

\pagestyle{fancy}
\fancyhf{}
\fancyhead[C]{図形と方程式(1)}
\renewcommand{\headrulewidth}{2pt}
\renewcommand{\footrulewidth}{0pt}

\begin{document}

\begin{tcolorbox}[title=数学\textcircled{1} 4- 2 AB]
$k$は実数の定数とする。\(x, y\)平面上に,2直線
\begin{align*}
l&:(k+2)x -(k-1)y -k -5 =0,\\
m&:x+2y-9=0
\end{align*}
がある.
\begin{enumerate}[(1)]
    \item 直線$l$は$k$の値にかかわらず定点Aを通る.Aの座標を求めよ.
    \item 2直線$l$,$m$が平行となる$k$の値,垂直となる$k$の値をそれぞれ求めよ.
    \item2直線$l$,$m$が平行となるとき,2直線間の距離を求めよ.
\end{enumerate}
\end{tcolorbox}

\end{document}
